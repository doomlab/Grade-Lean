\documentclass[man]{apa6}

\usepackage{amssymb,amsmath}
\usepackage{ifxetex,ifluatex}
\usepackage{fixltx2e} % provides \textsubscript
\ifnum 0\ifxetex 1\fi\ifluatex 1\fi=0 % if pdftex
  \usepackage[T1]{fontenc}
  \usepackage[utf8]{inputenc}
\else % if luatex or xelatex
  \ifxetex
    \usepackage{mathspec}
    \usepackage{xltxtra,xunicode}
  \else
    \usepackage{fontspec}
  \fi
  \defaultfontfeatures{Mapping=tex-text,Scale=MatchLowercase}
  \newcommand{\euro}{€}
\fi
% use upquote if available, for straight quotes in verbatim environments
\IfFileExists{upquote.sty}{\usepackage{upquote}}{}
% use microtype if available
\IfFileExists{microtype.sty}{\usepackage{microtype}}{}

% Table formatting
\usepackage{longtable, booktabs}
\usepackage{lscape}
% \usepackage[counterclockwise]{rotating}   % Landscape page setup for large tables
\usepackage{multirow}		% Table styling
\usepackage{tabularx}		% Control Column width
\usepackage[flushleft]{threeparttable}	% Allows for three part tables with a specified notes section
\usepackage{threeparttablex}            % Lets threeparttable work with longtable

% Create new environments so endfloat can handle them
% \newenvironment{ltable}
%   {\begin{landscape}\begin{center}\begin{threeparttable}}
%   {\end{threeparttable}\end{center}\end{landscape}}

\newenvironment{lltable}
  {\begin{landscape}\begin{center}\begin{ThreePartTable}}
  {\end{ThreePartTable}\end{center}\end{landscape}}

  \usepackage{ifthen} % Only add declarations when endfloat package is loaded
  \ifthenelse{\equal{\string man}{\string man}}{%
   \DeclareDelayedFloatFlavor{ThreePartTable}{table} % Make endfloat play with longtable
   % \DeclareDelayedFloatFlavor{ltable}{table} % Make endfloat play with lscape
   \DeclareDelayedFloatFlavor{lltable}{table} % Make endfloat play with lscape & longtable
  }{}%



% The following enables adjusting longtable caption width to table width
% Solution found at http://golatex.de/longtable-mit-caption-so-breit-wie-die-tabelle-t15767.html
\makeatletter
\newcommand\LastLTentrywidth{1em}
\newlength\longtablewidth
\setlength{\longtablewidth}{1in}
\newcommand\getlongtablewidth{%
 \begingroup
  \ifcsname LT@\roman{LT@tables}\endcsname
  \global\longtablewidth=0pt
  \renewcommand\LT@entry[2]{\global\advance\longtablewidth by ##2\relax\gdef\LastLTentrywidth{##2}}%
  \@nameuse{LT@\roman{LT@tables}}%
  \fi
\endgroup}


\ifxetex
  \usepackage[setpagesize=false, % page size defined by xetex
              unicode=false, % unicode breaks when used with xetex
              xetex]{hyperref}
\else
  \usepackage[unicode=true]{hyperref}
\fi
\hypersetup{breaklinks=true,
            pdfauthor={},
            pdftitle={Perceived Grading and Student Evaluation of Instruction},
            colorlinks=true,
            citecolor=blue,
            urlcolor=blue,
            linkcolor=black,
            pdfborder={0 0 0}}
\urlstyle{same}  % don't use monospace font for urls

\setlength{\parindent}{0pt}
%\setlength{\parskip}{0pt plus 0pt minus 0pt}

\setlength{\emergencystretch}{3em}  % prevent overfull lines


% Manuscript styling
\captionsetup{font=singlespacing,justification=justified}
\usepackage{csquotes}
\usepackage{upgreek}

 % Line numbering
  \usepackage{lineno}
  \linenumbers


\usepackage{tikz} % Variable definition to generate author note

% fix for \tightlist problem in pandoc 1.14
\providecommand{\tightlist}{%
  \setlength{\itemsep}{0pt}\setlength{\parskip}{0pt}}

% Essential manuscript parts
  \title{Perceived Grading and Student Evaluation of Instruction}

  \shorttitle{Perceived Grading and Student Evaluation}


  \author{Erin M. Buchanan\textsuperscript{1}, Becca N. Huber\textsuperscript{1, 2}, Arden Miller\textsuperscript{1}, David W. Stockburger\textsuperscript{3}, \& Marshall Beauchamp\textsuperscript{4}}

  \def\affdep{{"", "", "", "", ""}}%
  \def\affcity{{"", "", "", "", ""}}%

  \affiliation{
    \vspace{0.5cm}
          \textsuperscript{1} Missouri State University\\
          \textsuperscript{2} Idaho State University\\
          \textsuperscript{3} US Air Force Academy\\
          \textsuperscript{4} University of Missouri - Kansas City  }

  \authornote{
    \newcounter{author}
    A portion of this research was presented at the meeting of the
    Southwestern Psychological Association, April, 2009, San Antonio, TX.
    The authors would like to thank Melissa Fallone for comments on earlier
    drafts and Stephen Martin for his help with restructuring the data.

                                                                }


  \abstract{We analyzed student evaluations for 3,585 classes collected over 20
years to determine stability and evaluate the relationship of perceived
grading to global evaluations, perceived fairness, and appropriateness
of assignments. Using class as the unit of analysis, we found small
evaluation reliability when professors taught the same course in the
same semester, with much weaker correlations for differing courses.
Expected grade and grading related questions correlated with overall
evaluations of courses. Differences in course evaluations on expected
grades, grading questions, and overall grades were found between
full-time faculty and other types of instructors. These findings are
expanded to a model of grading type questions mediating the relationship
between expected grade and overall course evaluations with a moderating
effect of type of instructor.}
  \keywords{Student evaluation, teacher evaluation, perceived grading, reliability \\

    
  }





\usepackage{amsthm}
\newtheorem{theorem}{Theorem}
\newtheorem{lemma}{Lemma}
\theoremstyle{definition}
\newtheorem{definition}{Definition}
\newtheorem{corollary}{Corollary}
\newtheorem{proposition}{Proposition}
\theoremstyle{definition}
\newtheorem{example}{Example}
\theoremstyle{definition}
\newtheorem{exercise}{Exercise}
\theoremstyle{remark}
\newtheorem*{remark}{Remark}
\newtheorem*{solution}{Solution}
\begin{document}

\maketitle

\setcounter{secnumdepth}{0}



Student evaluations of professors are a typical practice, but their
validity and reliability has been disputed. The impact of student
evaluations on professor advancement can be great and often acts as a
deciding factor in professor promotion, demotion, coursework choice,
tenureship, or to inform access to certain funding opportunities. Some
suggest that there are variables that result in improving evaluations,
such as giving higher grades (Greenwald \& Gillmore, 1997; Isely \&
Singh, 2005; Krautmann \& Sander, 1999). Student evaluations are also
influenced by likability, attractiveness, and dress (Buck \& Tiene,
1989; Gurung \& Vespia, 2007; Hugh Feeley, 2002). Further, 20 years ago,
({\textbf{???}}) suggested 20 tongue-in-cheek tips in which professors
may bolster their evaluations from students. These suggestions have no
relationship with research supported instructional methods or further
learning retention among the student body, such as being a male
professor and only teaching only male students. In more recent research,
Boring, Ottoboni, \& Stark (2016) confirms that student evaluations of
teaching are biased against female instructors, and the authors conclude
student evaluations are more representative of the students' grading
expectations and biases rather than an evaluation of objective
instructional methods. All together, these findings elicit the argument
that student evaluations are not necessarily measuring whether the
instructional methods of professors are sound, rather student
evaluations of instruction are measuring whether or not the instructor
met the students' expectations of their performance in the classroom, in
addition to the instructor meeting pre-existing biases.

However, this finding does not imply that an instructor can simply raise
grades to meet expections (Centra, 2003; Marsh, 1987; Marsh \& Roche,
2000), instead one should consider the effect of \enquote{perceived
grading}. We operationally define perceived grading as the students'
perceptions of assignment appropriateness, grading fairness, and the
expected course grade at the time the evaluations are being completed.
STOPPED HERE

Social psychology theory would support that students with low perceived
grading may reduce cognitive dissonance and engage in ego defense by
giving low evaluations of professors who give them lower grades (Maurer,
2006), subsequently resulting in decreased validity and reliability of
the proposed construct, professor instruction. We argue both social
psychology theory and the evidence from student evaluations supports
that higher perceived grading can lead to better student evaluations of
instruction. For example, Salmons (1993) provided causal evidence of
lowered student evaluations due to expected grades. In her study of 444
students completing faculty evaluations at two separate points in a
semester, students who expected to get Fs significantly lowered their
evaluations while students who expected to receive As and Bs
significantly raised their evaluations (Salmons, 1993). This theory and
evidence from student evaluation leads us to further argue student
evaluations of professors are biased methods of data collection and
unrepresentative of the quality of the instructor and the instructional
methods used over the course of a semester.

Much of the literature on student evaluations involves diverse and
complex analyses (e.g., Marsh (1987)) and lacks social-psychological
theoretical guidance on human judgment. To expect that student
evaluations would not be influenced by expected grade would contradict a
long-standing history of social psychology research on cognitive
dissonance, attribution, and ego threat. As we know, failure threatens
the ego (Miller, 1985) and motivates us to find rationales to defend the
ego. Further, ({\textbf{???}}) found guilt as a significant correlate of
dissonance which may be illuminated in this study by the guilt of
underperforming from a student's own expectations. Failing students, or
those performing below personal expectations, would be expected to
defend their ego by attributing low grades to poor teaching or unfair
evaluation practices (Maurer, 2006). One common strategy involves
diminishing the value of the activity (Miller \& Klein, 1989), which
would result in lowered perceived value of a course.

Similarly, Cognitive Dissonance Theory (Festinger, 1957) predicts that
people who experience poor performance but perceive themselves as
competent will experience dissonance, of which they can reduce through
negative evaluations of the instruction (Maurer, 2006). Attribution
research (Weiner, 1992) also supports the argument that among low
achievement motivation students, failure is associated with external
attributions for cause, and the most plausible external attribution for
the student in the evaluation context is the quality of instruction and
grading practices. Although arguments regarding degree of influence are
reasonable, the position that they are not affected is inconsistent with
existing and established theory. Thus, it is not surprising that the
majority of faculty perceive student evaluations to be biased by
perceived grading and course choice (Marsh, 1987).

Considerable research has been conducted in support of widely
distributed evaluation systems. Centra (2003) reported that in a study
of 9,194 class averages using the Student Instructional Support, the
relationship between expected grades and global ratings was only .20. He
further argued that when variance due to perceived learning outcomes was
regressed from the global evaluation, the effect of expected grades was
eliminated. However, a student's best assessment of \enquote{perceived
learning outcome} is their expected grade, and thus, these should be
highly correlated. When perceived learning is regressed from the global
evaluations, it is not surprising that suppression effects would
eliminate or could even reverse the correlation between expected grade
and global evaluation. In general, there are several reasons why the
relationship of expected grade to global evaluations is suppressed. For
example, faculty ratings are generally very high on average
(i.e.~quality instructors are hired), which restricts variation; thus,
weakening their reliability as a measure of professor attributes. This
restriction in range suppresses correlation.

Marsh (1987) argued that the individual is also not the proper unit of
analysis because such analyses could suggest false findings related to
individual differences in students. Therefore, he argued the use of
class as the suggested unit of analysis (Marsh, 1987). We agree, both
for his reasoning and because analyses with individual ratings can mask
significant relationships as well ({\textbf{???}}, {\textbf{???}}).
Individual differences in expectancy will attenuate the correlation less
when class average is used as the unit of analysis. To the extent that
the same class average would be expected across all courses, an
assumption we challenged, the class average for expected grade is a good
measure of perceived grading as an instructor attribute. Course quality,
not individual attributes of students, is what we attempted to assess
when we used student evaluations of courses. Several studies provide
support that when class is the unit of analysis, expected grade is a
more significant biasing factor in student evaluations (Blackhart,
Peruche, DeWall, \& Joiner, 2006; DuCette \& Kenney, 1982; Ellis, Burke,
Lomire, \& McCormack, 2003).

Additionally, Blackhart et al. (2006) analyzed 167 psychology classes in
a multiple regression analysis with class as the unit of analysis and
found the two most significant predictors of instructor ratings were
average grade given by the instructor and instructor status (TA or rank
of faculty). Given the restricted number of classes, the power of the
analysis was limited. However, in addition to the concern regarding the
relationship between grades and global course evaluations, it was found
that TAs were rated more highly than ranked faculty. This finding raises
additional questions on validity of student evaluation regarding
instructional quality (Blackhart et al., 2006). We must either accept
that the least trained and qualified instructors are actually better
teachers, or we must believe this result suggests student evaluations
have given us false information on the quality of instruction via their
perceptions of grading.

DuCette \& Kenney (1982) provided further evidence that using course as
a unit of analysis increased the correlation between expected grade and
other course ratings. Within specific groupings of classes, these
correlations ranged from .23 to .53. However, two factors limited the
level of their relationships. First, the classes used were all upper
division courses and graduate courses. Secondly, over 90\% of the
students in these classes expected an A or a B. Consequently, the
correlations between expected grade and global course ratings would be
reduced due to the absence of variation in expected grades. Similarly,
Ellis et al. (2003) found a correlation of .35 between average course
grade and average rating of the instructor in 165 classes during a
two-year period. Although, these studies did not consider the predictive
relationship for instructors across different courses and semesters,
which was one aim of the current study.

It is pertinent to note that different disciplines and subject areas
have diverse GPA standards, and students have differing grade and
workload expectations in different courses, as well. For example, an
instructor in Anatomy giving a 3.00 GPA might be considered lenient
while an Education instructor giving a 3.25 GPA might be considered hard
(examples for illustration only). To have a valid measure of workload
and leniency factors, correlations should be conducted with varied
teachers of the same course. Further, different populations take courses
in different disciplines, resulting in potential population differences
between anatomy classes and education classes, which could create or
mask findings. Hence, analysis of these correlations within the same
discipline and course would be expected to strengthen the relationship
between expected grades and quality measures, offering more valid
results.

Further, in most studies of student evaluations, reliability is
established through internal consistency reliability. However, this form
of reliability is confounded with halo effects (i.e.~a cognitive bias
that influences ratings based on an overall perception of the person
teaching, rather than the individual components of the course), and
tells only whether the individual responding to the questions is
consistent and reliable. By having many different classes for the same
instructor, we can establish the reliability of ratings across the same
and different courses during the same and different semesters. As a
result, we should be able to deduce if student ratings can be considered
a valid measure of an instructor's teaching skills if they are or are
not able to reliably differentiate instructors within the same course
across different semesters.

If ratings are, in fact, valid measures of instructor attributes, it
should be expected that ratings would have some stability across
semester and specific course taught. If variation were due to instructor
attributes and not the course they are assigned, we would expect ratings
to be most stable across two different courses during the same semester.
We would expect these correlations to decline somewhat for the same
course in a different semester, since faculty members may improve or
decline with experience. However, if they are reliable and stable enough
to use in making choices about retention, their stability should be
demonstrated across different semesters, as well. Therefore, in the
current study, we first sought to establish reliability of ratings for
the instructors across courses and semesters.

The current study used data collected over a 20-year period to allow for
more powerful analyses, with such analyses occurring within many
sections of the same course at the same university. After examining
reliability, we sought to show that items on instructor evaluations were
positively correlated for undergraduate students, demonstrating that
overall course evaluations are related to the perceived grading of the
students. We also expected correlations to be substantially higher than
those obtained by previous researchers who used individual students as
their unit of analysis, since we used the course as the unit of
analysis. Next, we examined if rating differences across these questions
were found between types of instructors compared to full-time faculty,
such as teaching-assistants and per-course faculty. The presumption of
university hiring requirements that include a terminal degree for
regular faculty is that better-trained faculty will be more effective
teachers. Therefore, if student evaluations are a valid measure,
better-trained, full-time faculty should receive higher ratings than
per-course instructors and teaching assistants. However, existing
literature appears to contradict this expectation (Blackhart et al.,
2006). Given these differences, we proposed and examined a moderated
mediation analysis to portray the expected relationship of the variables
across instructor type.

\section{Method}\label{method}

The archival study was conducted using data from the psychology
department at a large Midwestern public university. We used data from
4313 undergraduate, 397 mixed-level undergraduate, and 687 graduate
psychology classes taught from 1987 to 2016 that were evaluated by
students using the same 15-item instrument. The graduate courses were
excluded from analyses due to the ceiling effects on expected grades.
Faculty followed set procedures in distributing scan forms no more than
two weeks before the conclusion of the semester. A student was assigned
to collect the forms and deliver them to the departmental secretary. The
instructor was required to leave the room while students completed the
forms.

We focused upon the five items, which seemed most pertinent to the
issues of perceived grading and evaluation. We were most interested in
how grades related to global course evaluation and grading/assignment
evaluations. These items were presented with a five-point scale from 1
(\emph{strongly disagree}) to 5 (\emph{strongly agree}):

\begin{verbatim}
1. The overall quality of this course was among the top 20% of those I have taken. 
2. The examinations were representative of the material covered in the assigned readings and class lectures. 
3. The instructor used fair and appropriate methods in the determination of grades. 
4. The assignments and required activities in this class were appropriate. 
5. What grade do you expect to receive in this course? (A = 5, B, C, D, F = 1).
\end{verbatim}

\section{Results}\label{results}

All data were checked for course coding errors, and type of instructor
was coded as teaching assistant, per-course faculty, instructors, and
tenure-track faculty. This data was considered structured by instructor;
therefore, all analyses below were coded in \emph{R} using the
\emph{nlme} package (Pinheiro, Bates, Debroy, Sarkar, \& Team, 2017) to
control for correlated error of instructor as a random intercept in a
multilevel model. The overall dataset was screened for normality,
linearity, homogeneity, and homoscedasticity using procedures from
Tabachnick \& Fidell (2012). Data generally met assumptions with a
slight skew and some heterogeneity. This data was not screened for
outliers because it was assumed that each score was entered correctly
from student evaluations. The complete set of all statistics can be
found online at \url{http://osf.io/jdpfs}. This page also includes the
manuscript written online with the statistical analysis using the
\emph{papaja} package ({\textbf{???}}) for interested
researchers/reviewers.

\subsection{Reliability of Instructor Scores
DONE}\label{reliability-of-instructor-scores-done}

\begin{table}[tbp]
\begin{center}
\begin{threeparttable}
\caption{\label{tab:rel-table}Correlations for Instructor, Semester, and Course Combinations}
\small{
\begin{tabular}{lllccccc}
\toprule
Instructor & Semester & Course & $b$ & $SE$ & $df$ & $t$ & $p$\\
\midrule
Different Instructor & Different Semester & Different Course & -.001 & .000 & 10144295 & -3.580 & .013\\
Different Instructor & Same Semester & Different Course & .006 & .002 & 152801 & 2.906 & .048\\
Different Instructor & Different Semester & Same Course & .008 & .001 & 517353 & 6.236 & .027\\
Different Instructor & Same Semester & Same Course & .054 & .010 & 6265 & 5.402 & < .001\\
Same Instructor & Different Semester & Different Course & -.038 & .003 & 108849 & -13.130 & < .001\\
Same Instructor & Same Semester & Different Course & .095 & .020 & 1872 & 4.659 & < .001\\
Same Instructor & Different Semester & Same Course & .090 & .004 & 55057 & 21.769 & < .001\\
Same Instructor & Same Semester & Same Course & .446 & .023 & 1401 & 19.631 & < .001\\
\bottomrule
\end{tabular}
}
\end{threeparttable}
\end{center}
\end{table}

Reliability of ratings of instructors can be inferred by the consistency
of ratings across courses and semester, assuming that we infer there is
a stable good/poor instructor attribute and that these multiple
administrations of the same question are multiple assessments of that
attribute. A file was created with all possible course pairings for
every instructor, semester, and course combination. Therefore, this
created eight possible combinations of matching v. no match for
instructor by semester by course. Multilevel models were used to
calculate correlations on each fo the eight combinations controlling for
response size for both courses (i.e., course 1 number of ratings and
course 2 number of ratings) and random intercepts for instructor(s).
Correlations were calculated separately for each question, however, the
overall pattern of the data was the same for each of the eight
combinations, and these were averaged for Table @ref:(tab:rel-table).
The complete set of all correlations can be found online. Given the
large sample size can bias statistical significance, we focused on the
size of the correlations. The correlations were largest for the same
instructor in the same semester and course, followed by the same
instructor in the same semester with a different course and the same
instructor in a different semester with the same course. The first shows
that scores are somewhat reliable (i.e., \emph{r}s \textasciitilde{}
.45) for instructors teaching two or more of the same class at the same
time. The correlations within instructor then drop to \emph{r}s
\textasciitilde{} .09 for the same semester or same course. All other
correlations are nearly zero, with the same semester, same course, and
different instructor as the next largest at \emph{r}s \textasciitilde{}
.05. Given these values are still low for traditional reliability
standards, these results may indicate that student demand
characteristics or course changes impact instructor ratings.

\subsection{Correlations of Evaluation Questions
DONE}\label{correlations-of-evaluation-questions-done}

\begin{table}[tbp]
\begin{center}
\begin{threeparttable}
\caption{\label{tab:correlation-table}t Statistics for Inter-item Relationship}
\begin{tabular}{lcccccl}
\toprule
Coefficient & \multicolumn{1}{c}{$pr$} & \multicolumn{1}{c}{$b$} & \multicolumn{1}{c}{$SE$} & \multicolumn{1}{c}{$df$} & \multicolumn{1}{c}{$t$} & \multicolumn{1}{c}{$p$}\\
\midrule
Overall to Exams & .637 & .828 & .014 & 4447 & 60.813 & < .001\\
Overall to Fair & .606 & .903 & .016 & 4447 & 57.837 & < .001\\
Overall to Assignments & .675 & .999 & .016 & 4447 & 63.251 & < .001\\
Overall to Expected Grade & .344 & .597 & .022 & 4447 & 27.167 & < .001\\
Exams to Fair & .655 & .751 & .012 & 4447 & 61.387 & < .001\\
Exams to Assignments & .615 & .700 & .014 & 4447 & 50.425 & < .001\\
Exams to Expected Grade & .311 & .416 & .018 & 4447 & 23.066 & < .001\\
Fair to Assignments & .720 & .715 & .011 & 4447 & 63.912 & < .001\\
Fair to Expected Grade & .375 & .438 & .016 & 4447 & 27.865 & < .001\\
Assignments to Expected Grade & .344 & .404 & .015 & 4447 & 26.913 & < .001\\
\bottomrule
\end{tabular}
\end{threeparttable}
\end{center}
\end{table}

Table \ref{tab:correlation-table} presents the inter-correlations for
the five relevant evaluation questions using instructor as a random
intercept in a multilevel model with evaluation sample size as an
adjustor variable. The partial correlation (\emph{pr}) is the
standardized coefficient from the multilevel model analysis between
items while adjusting for sample size and random effects of instructor.
The raw coefficient \emph{b}, standard error, and significance
statistics are also provided. We found class expected grade was related
to class overall rating, exams reflecting the material, grading
fairness, and appropriateness of assignments; however, these partial
correlations were approximately half of all other pairwise correlations.
The correlations between grading related items were high, representing
some consistency in evaluation, as well as the overall course evaluation
to grading questions.

\subsection{Moderated Mediation}\label{moderated-mediation}

\begin{table}[tbp]
\begin{center}
\begin{threeparttable}
\caption{\label{tab:table-mod-med}t Statistics for Moderated Mediation}
\begin{tabular}{llccccc}
\toprule
DV & \multicolumn{1}{c}{IV} & \multicolumn{1}{c}{$b$} & \multicolumn{1}{c}{$SE$} & \multicolumn{1}{c}{$df$} & \multicolumn{1}{c}{$t$} & \multicolumn{1}{c}{$p$}\\
\midrule
Overall Course & Expected Grade & 0.493 & 0.102 & 4336 & 4.857 & < .001\\
Overall Course & Teaching Assistant & 0.114 & 0.085 & 191 & 1.345 & .180\\
Overall Course & Per-Course & -0.102 & 0.116 & 191 & -0.880 & .380\\
Overall Course & Instructor & 0.096 & 0.081 & 191 & 1.187 & .237\\
Overall Course & EG X TA & 0.126 & 0.126 & 4336 & 0.996 & .319\\
Overall Course & EG X PC & 0.304 & 0.115 & 4336 & 2.637 & .008\\
Overall Course & EG X IN & 0.049 & 0.105 & 4336 & 0.464 & .643\\
Average Grading & Expected Grade & 0.416 & 0.062 & 4336 & 6.667 & < .001\\
Average Grading & Teaching Assistant & -0.023 & 0.047 & 191 & -0.492 & .623\\
Average Grading & Per-Course & -0.132 & 0.063 & 191 & -2.096 & .037\\
Average Grading & Instructor & -0.083 & 0.044 & 191 & -1.860 & .064\\
Average Grading & EG X TA & 0.111 & 0.078 & 4336 & 1.428 & .153\\
Average Grading & EG X PC & 0.117 & 0.071 & 4336 & 1.642 & .101\\
Average Grading & EG X IN & -0.056 & 0.064 & 4336 & -0.870 & .384\\
Overall Course & Expected Grade & -0.024 & 0.077 & 4332 & -0.313 & .755\\
Overall Course & Teaching Assistant & 0.142 & 0.048 & 191 & 2.936 & .004\\
Overall Course & Per-Course & 0.065 & 0.063 & 191 & 1.028 & .305\\
Overall Course & Instructor & 0.198 & 0.045 & 191 & 4.388 & < .001\\
Overall Course & Average Grading & 0.000 & 0.000 & 4332 & 1.768 & .077\\
Overall Course & EG X TA & -0.126 & 0.098 & 4332 & -1.283 & .200\\
Overall Course & EG X PC & 0.206 & 0.091 & 4332 & 2.271 & .023\\
Overall Course & EG X IN & 0.173 & 0.080 & 4332 & 2.164 & .031\\
Overall Course & AG X TA & 0.216 & 0.103 & 4332 & 2.107 & .035\\
Overall Course & AG X PC & -0.081 & 0.099 & 4332 & -0.821 & .412\\
Overall Course & AG X IN & -0.142 & 0.087 & 4332 & -1.634 & .102\\
\bottomrule
\end{tabular}
\end{threeparttable}
\end{center}
\end{table}

\begin{table}[tbp]
\begin{center}
\begin{threeparttable}
\caption{\label{tab:table-med-split}t Statistics for Individual Mediations}
\begin{tabular}{lllccccc}
\toprule
Group & \multicolumn{1}{c}{DV} & \multicolumn{1}{c}{IV} & \multicolumn{1}{c}{$b$} & \multicolumn{1}{c}{$SE$} & \multicolumn{1}{c}{$df$} & \multicolumn{1}{c}{$t$} & \multicolumn{1}{c}{$p$}\\
\midrule
Teaching Assistant & Overall Course & Expected Grade & 0.510 & 0.092 & 219 & 5.534 & < .001\\
Teaching Assistant & Average Grading & Expected Grade & 0.407 & 0.049 & 219 & 8.326 & < .001\\
Teaching Assistant & Overall Course & Expected Grade & -0.010 & 0.077 & 218 & -0.126 & .900\\
Teaching Assistant & Overall Course & Average Grading & 1.265 & 0.084 & 218 & 15.017 & < .001\\
Per-Course & Overall Course & Expected Grade & 0.605 & 0.071 & 425 & 8.536 & < .001\\
Per-Course & Average Grading & Expected Grade & 0.505 & 0.040 & 425 & 12.640 & < .001\\
Per-Course & Overall Course & Expected Grade & -0.109 & 0.051 & 424 & -2.163 & .031\\
Per-Course & Overall Course & Average Grading & 1.426 & 0.049 & 424 & 28.991 & < .001\\
Instructor & Overall Course & Expected Grade & 0.836 & 0.054 & 504 & 15.511 & < .001\\
Instructor & Average Grading & Expected Grade & 0.562 & 0.035 & 504 & 15.967 & < .001\\
Instructor & Overall Course & Expected Grade & 0.194 & 0.044 & 503 & 4.375 & < .001\\
Instructor & Overall Course & Average Grading & 1.144 & 0.045 & 503 & 25.230 & < .001\\
Tenure Track & Overall Course & Expected Grade & 0.537 & 0.027 & 3185 & 19.817 & < .001\\
Tenure Track & Average Grading & Expected Grade & 0.359 & 0.017 & 3185 & 20.722 & < .001\\
Tenure Track & Overall Course & Expected Grade & 0.142 & 0.021 & 3184 & 6.891 & < .001\\
Tenure Track & Overall Course & Average Grading & 1.097 & 0.020 & 3184 & 56.152 & < .001\\
\bottomrule
\end{tabular}
\end{threeparttable}
\end{center}
\end{table}

We proposed a mediation relationship between expected grade, perceived
grading, and overall course grades that varies by instructor type.
Figure 1 demonstrates the predicted relationship between these
variables. We hypothesized that expected course grade would impact the
overall course rating, but this relationship would be mediated by the
perceived grading in the course, which was calculated by averaging
questions about exams, fairness of grading, and assignments. Therefore,
as students expected to earned higher grades, their perception and
ratings of the grading would increase, thus, leading to higher overall
course scores. This relationship was tested using traditional and newer
approaches to mediation (Baron \& Kenny, 1986; Hayes, 2017). All
categorical interactions were compared to ranked faculty. Each step of
the model is described below. Because significant interactions were
found, we calculated each group separately (Figure 1) to portray these
differences in path coefficients. Tables of t values for the overall and
separated analyses are available at \url{http://osf.io/jdpfs}.

\subsubsection{C Path}\label{c-path}

First, expected grade was used to predict the overall rating of the
course, along with the interaction of type of instructor and expected
grade. The expected grade positively predicted overall course rating, p
\textless{} .001, wherein higher expected grades was related to higher
overall ratings for the course (b = 0.39). A significant interaction
between type and expected grade rating was found for instructors versus
faculty. In looking at Figure 1, we find that instructors (b = 0.56)
have a stronger relationship between expected grade and overall course
rating than faculty (b = 0.39, interaction p = .020), while per-course
(b = 0.41, interaction p = .621) and teaching assistants (b = 0.71,
interaction p = .068) were not significantly different than faculty on
the c path coefficient.

\subsubsection{A Path}\label{a-path}

Expected grade was then used to predict the average of the grading
related questions, along with the interaction of type of instructor.
Higher expected grades were related to higher ratings of appropriating
grading (b = 0.21, p \textless{} .001), and a significant interaction of
faculty and all three other instructor types emerged: teaching
assistants (p = .001), per-course faculty (p = .001), and instructors (p
\textless{} .001). As seen in Figure 1, faculty (b = 0.21) have a much
weaker relationship between expected grade and average ratings of
grading than teaching assistants (b = 0.55), per-course (b = 0.41), and
instructors (b = 0.45).

\subsubsection{B and C' Paths}\label{b-and-c-paths}

In the final model, expected grade, average ratings of grading, and the
two-way interactions of these two variables with type were used to
predict overall course evaluation. Average rating of grading was a
strong significant predictor of overall course rating (b = 1.10, p
\textless{} .001), indicating that a perception of fair grading was
related positively to overall course ratings. An interaction between
per-course faculty and fair grading emerged, p \textless{} .001, wherein
faculty (b = 1.10) had a less positive relationship than per-course (b =
1.28), while teaching assistants (b = 1.37, interaction p = .071) and
instructors (b = 1.16, interaction p = .187) were not significantly
different coefficients. The relationship between expected grade and
overall course rating decreased from the original model (b = 0.16, p
\textless{} .001). However, the interaction between this path and
per-course (p \textless{} .001) and instructors (p = .041) versus
faculty was significant, while faculty versus teaching assistants' paths
were not significantly different (p = .133). Faculty relationship
between expected grade and overall course scoring, while accounting for
ratings of grading was stronger (b = 0.16) than instructors (b = 0.04)
and per-course (b = -0.10), but not that of teaching assistants (b =
-0.04).

\subsubsection{Mediation Strength}\label{mediation-strength}

We then analyzed the indirect effects (i.e.~the amount of mediation) for
each type of instructor separately, using both the Aroian version of the
Sobel test (Baron \& Kenny, 1986), as well as bootstrapped samples to
determine the 95\% confidence interval of the mediation (Preacher \&
Hayes, 2008; Hayes, 2017) due to the criticisms of Sobel. For confidence
interval testing, we ran 5,000 bootstrapped samples examining the
mediation effect and interpreted that the mediation was different from
zero if the confidence interval did not include zero. For teaching
assistants, we found mediation significantly greater than zero, indirect
= 0.74 (SE = 0.14), Z = 5.15, p \textless{} .001, 95\% CI{[}0.48,
1.02{]}. Additionally, per-course faculty showed mediation between
expected grade and overall course rating, indirect = 0.52 (SE = 0.09), Z
= 6.06, p \textless{} .001, 95\% CI{[}0.36, 0.73{]}, and instructors
showed a similar indirect mediation effect, indirect = 0.53 (SE = 0.07),
Z = 7.31, p \textless{} .001, 95\% CI{[}0.40, 0.66{]}. Last, faculty
showed the smallest mediation effect, indirect = 0.23 (SE = 0.02), Z =
8.71, p \textless{} .001, 95\% CI{[}0.19, 0.28{]}, wherein the
confidence interval did not include zero, but also did not overlap with
any other instructor group.

\section{Discussion}\label{discussion}

The findings support the model that student evaluations of Psychology
faculty are related to what one might consider leniency (i.e., overall
average scores of B) in grading through perceptions of assignment
appropriateness, grading fairness, and the expected course grade. This
position is supported both in the strong relationships between expected
grade and global ratings by the evidence that greater training and
experience is related to poorer evaluations, lower expected grades, and
lower relationships between grading and evaluations. Faculty received
lower scores than teaching assistants in every category and often lower
scores than per-course faculty, but not instructors. Mediation analyses
showed that expected grade is positively related to overall course
ratings, although this relationship is mediated by the perceived grading
in the course. Therefore, as students have higher expected grades, the
perceived grading scores increase, and the overall course score also
increases. Moderation of this mediation effect indicated differences in
the strength of the relationships between expected grade, grading
questions, and overall course rating, wherein faculty generally had
weaker relationships between these variables.

Because the study was not experimental, causal conclusions from this
study alone need to be limited. However, Salmons (1993) provides some
evidence of the causal direction of student ratings of instructors and
expected grades. She had 444 students complete faculty evaluations after
3-4 weeks of classes, and again after 13 weeks. Students who expected to
get Fs significantly lowered their evaluations while students who
expected to receive As and Bs significantly raised their evaluations.

It is compelling that the correlations suggest that we can do a better
job of understanding global ratings, perception of exams, fairness, and
appropriateness of assignments based upon the grade students expect as
compared to relating these ratings using ratings for the same course in
a different semester or ratings for a different course in the same
semester for instructor (i.e., correlations between items in the same
semester are higher than reliability estimates across the board). It is
very likely these correlations with expected grade are suppressed by the
loading of scores at the high end of the scale for course ratings and
expected grade. Generally, evaluation items reflect scores at the high
end of the 1-5 scale (see Table 3) even when items are intentionally
constructed to move evaluators from the ends. The item, \enquote{The
overall quality of this course was among the top 20\% of those I have
taken,} is conspicuously designed to move subjects away from the top
rating. Yet, average global ratings remain about a 4.00. The grade
expectation average was approximately 4.00, which relates to a B average
or 3.00 GPA.

One way of establishing convergent validity would be a finding that
better trained and more experienced teachers get higher ratings than
less well trained instructors. If the measure were valid, we would
expect that regular faculty and full time instructors would get higher
ratings than per course faculty and teaching assistants. To argue
otherwise is to challenge the merits of higher education units with a
faculty of professors with doctoral status. If the university were a
researcher powerhouse where faculty research is emphasized over teaching
and graduate assistants are admitted from the highest ranks of
undergraduates, the finding that teaching assistants and per course
faculty get higher ratings might be less of a challenge to the validity
of these ratings. However, the university at which the data were
collected is a non-doctoral program with greater emphasis on teaching
and moderate emphasis on research, and teaching assistants are master's
candidates with less substantial admission expectations than doctoral
programs. Hence, these findings challenge the convergent validity of the
teaching evaluations.

Like most studies in this area, a major limitation is the absence of an
independent measure of learning. Of course, this limitation is based
upon the belief that the goal is to create educated persons, not just
satisfied consumers. Even when common tests are used, these are invalid
if the instructors are aware of the course content. Teachers seeking
high evaluations are able to improve their ratings and scores by
directly addressing the content of the specific test items. ETS now
allows faculty who administer Major Field Tests to access the specific
items which thereby invalidates it as a measure for these purposes.
Ultimately, answering questions about the validity of student
evaluations is a daunting task without such measures.

Evidence suggests that student evaluations are influenced by likability,
attractiveness, and dress (Buck \& Tiene, 1989; Gurung \& Vespia, 2007;
Hugh Feeley, 2002) in addition to leniency and low demands (Greenwald \&
Gillmore, 1997). One must question whether a factor like instructor
warmth, which relates to student evaluation (Best \& Addison, 2000), is
really fitting to the ultimate purposes of a college education. In a
unique setting where student assignments to courses were random and
common tests were used, Carrell \& West (2010) demonstrated that
teaching strategies that enhanced student evaluations led to poorer
performance in subsequent classes. With the sum of invalid variance from
numerous factors being potentially high, establishment of a high
positive relationship to independent measures of achievement is
essential to the acceptance of student evaluations as a measure of
teaching quality.

Perception of the influence of leniency on teacher evaluations is far
more detrimental to the quality of education than the biased evaluations
themselves. It is unlikely that good teachers, even if more challenging,
will get bad evaluations (i.e.~evaluations where the majority of
students rate the course poorly). Good teachers are rarely losing their
positions due to low quality evaluations. But Marsh (1987) found that
faculty perceives evaluations to be biased based upon course difficulty
(72\%), expected grade (68\%), and course workload (60\%). If one's goal
is high merit ratings and teaching awards, and the most significant
factor is student evaluations of teaching, then putting easier and
low-level questions on the test, adding more extra credit, cutting the
project expectations, letting students off the hook for missing
deadlines, and boosting borderline grades would all be likely strategies
for boosting evaluations.

Effective teachers will get positive student ratings even when they have
high expectations and do not inflate grades. But, many excellent
teachers will score below average. It is maladaptive to try to increase
a 3.90 global rating to a 4.10, because it often requires that the
instructor try to emphasize avoidance of the lowest rating (1.00)
because these low ratings in a skewed distribution have in inordinate
influence on the mean. This effort of competing against the norms is
likely to lead to grade inflation and permissiveness for the least
motivated and most negligent students. Some researchers (Ellis et al.,
2003; Greenwald \& Gillmore, 1997) argue that student evaluations of
instruction should be adjusted on the basis of grades assigned. However,
there are problems with such an approach. The regression Betas are
likely to differ based upon course and many other factors. In our
research and in research by DuCettte and Kenney (1982), substantial
variation in correlations was found across different course sets.
Establishing valid adjustments would be problematic at best. Further,
such an approach would punish instructors when they happen to get an
unusually intelligent and motivated class (or teach an honors class) and
give students the grades they deserve. Student evaluations are not a
proper motivational factor for instructors in grade assignment, whether
it is to inflate or deflate grades.

It would seem nearly impossible to eliminate invalid bias in student
ratings of instruction. Yet, they may tell us a teacher is ineffective
when the majority give poor ratings. It is the normative, competitive
use that makes student evaluations of teaching subject to problematic
interpretation. This finding is especially critical in light of recent
research that portrays that student evaluations are largely biased
against female teachers, and that student bias in evaluation is related
to course discipline and student gender (Boring et al., 2016). Boring et
al. (2016) also examine the difficulty in adjusting faculty evaluation
for bias and determined that the complex nature of ratings makes
unbiased evaluation nearly impossible. Stark \& Freishtat (2014) further
explain that evaluations are often negatively related to more objective
measures of teaching effectiveness, and biased additionally by perceived
attractiveness and ethnicity. In line with the current paper, he
suggests dropping overall teaching effectiveness or value of the course
type questions because they are influenced by many variables unrelated
to actual teaching. Last, they suggest the distribution and response
rate of the data are critical information, and this point becomes
particularly important when recent research shows that online
evaluations of teaching experience a large drop in response rates
(Stanny \& Arruda, 2017). Our study contributes to the literature of how
student evaluations are a misleading and unsuccessful measure of
teaching effectiveness, especially focusing on reliability and the
impact of grading on overall questions. We conclude that it may be
possible to manipulate these values by lowering teaching standards,
which implies that high stakes hiring and tenure decisions should
probably follow the advice of Palmer, Bach, \& Streifer (2014) or
Stanny, Gonzalez, \& McGowan (2015) in implementing teaching portfolios
and syllabus review, particularly because a recent meta-analysis of
student evaluations showed they are unrelated to student learning (Uttl,
White, \& Gonzalez, 2017).

\newpage

\section{References}\label{references}

\setlength{\parindent}{-0.5in} \setlength{\leftskip}{0.5in}

\hypertarget{refs}{}
\hypertarget{ref-Baron1986}{}
Baron, R. M., \& Kenny, D. A. (1986). The moderator-mediator variable
distinction in social psychological research: Conceptual, strategic, and
statistical considerations. \emph{Journal of Personality and Social
Psychology}, \emph{51}(6), 1173--1182.
doi:\href{https://doi.org/10.1037//0022-3514.51.6.1173}{10.1037//0022-3514.51.6.1173}

\hypertarget{ref-Best2000}{}
Best, J. B., \& Addison, W. E. (2000). A preliminary study of perceived
warmth of professor and student evaluations. \emph{Teaching of
Psychology}, \emph{27}(1), 60--62. Retrieved from
\url{http://psycnet.apa.org/record/2000-07173-018}

\hypertarget{ref-Blackhart2006}{}
Blackhart, G. C., Peruche, B. M., DeWall, C. N., \& Joiner, T. E.
(2006). Factors influencing teaching evaluations in higher education.
\emph{Teaching of Psychology}, \emph{33}(1), 37--39.
doi:\href{https://doi.org/10.1207/s15328023top3301_9}{10.1207/s15328023top3301\_9}

\hypertarget{ref-Boring2016}{}
Boring, A., Ottoboni, K., \& Stark, P. (2016). Student Evaluations of
Teaching (Mostly) Do Not Measure Teaching Effectiveness.
\emph{ScienceOpen Research}.
doi:\href{https://doi.org/10.14293/S2199-1006.1.SOR-EDU.AETBZC.v1}{10.14293/S2199-1006.1.SOR-EDU.AETBZC.v1}

\hypertarget{ref-Buck1989}{}
Buck, S., \& Tiene, D. (1989). The Impact of Physical Attractiveness,
Gender, and Teaching Philosophy on Teacher Evaluations. \emph{The
Journal of Educational Research}, \emph{82}(3), 172--177.
doi:\href{https://doi.org/10.1080/00220671.1989.10885887}{10.1080/00220671.1989.10885887}

\hypertarget{ref-Carrell2010}{}
Carrell, S. E., \& West, J. E. (2010). Does Professor Quality Matter?
Evidence from Random Assignment of Students to Professors. \emph{Journal
of Political Economy}, \emph{118}(3), 409--432.
doi:\href{https://doi.org/10.1086/653808}{10.1086/653808}

\hypertarget{ref-Centra2003}{}
Centra, J. A. (2003). Will teachers recieve higher student evaluations
by giving higher grades and less course work? \emph{Research in Higher
Education}, \emph{44}(5), 495--518.
doi:\href{https://doi.org/10.1023/A:1025492407752}{10.1023/A:1025492407752}

\hypertarget{ref-DuCette1982}{}
DuCette, J., \& Kenney, J. (1982). Do grading standards affect student
evaluations of teaching? Some new evidence on an old question.
\emph{Journal of Educational Psychology}, \emph{74}(3), 308--314.
doi:\href{https://doi.org/10.1037/0022-0663.74.3.308}{10.1037/0022-0663.74.3.308}

\hypertarget{ref-Ellis2003}{}
Ellis, L., Burke, D. M., Lomire, P., \& McCormack, D. R. (2003). Student
Grades and Average Ratings of Instructional Quality: The Need for
Adjustment. \emph{The Journal of Educational Research}, \emph{97}(1),
35--40.
doi:\href{https://doi.org/10.1080/00220670309596626}{10.1080/00220670309596626}

\hypertarget{ref-Festinger1957}{}
Festinger, L. (1957). \emph{A Theory of Cognitive Dissonance}. Stanford,
CA: Stanford University Press. Retrieved from
\href{https://books.google.com/books?hl=en\%7B/\&\%7Dlr=\%7B/\&\%7Did=voeQ-8CASacC\%7B/\&\%7Doi=fnd\%7B/\&\%7Dpg=PA1\%7B/\&\%7Ddq=A+theory+of+cognitive+dissonance\%7B/\&\%7Dots=9y58Kxq9Bz\%7B/\&\%7Dsig=k7EkRmTqSooXrWutB7rga7FlDBs\%7B/\#\%7Dv=onepage\%7B/\&\%7Dq=A\%20theory\%20of\%20cognitive\%20dissonance\%7B/\&\%7Df=false}{https://books.google.com/books?hl=en\{\textbackslash{}\&\}lr=\{\textbackslash{}\&\}id=voeQ-8CASacC\{\textbackslash{}\&\}oi=fnd\{\textbackslash{}\&\}pg=PA1\{\textbackslash{}\&\}dq=A+theory+of+cognitive+dissonance\{\textbackslash{}\&\}ots=9y58Kxq9Bz\{\textbackslash{}\&\}sig=k7EkRmTqSooXrWutB7rga7FlDBs\{\textbackslash{}\#\}v=onepage\{\textbackslash{}\&\}q=A theory of cognitive dissonance\{\textbackslash{}\&\}f=false}

\hypertarget{ref-Greenwald1997}{}
Greenwald, A. G., \& Gillmore, G. M. (1997). Grading leniency is a
removable contaminant of student ratings. \emph{American Psychologist},
\emph{52}(11), 1209--1217.
doi:\href{https://doi.org/10.1037/0003-066X.52.11.1209}{10.1037/0003-066X.52.11.1209}

\hypertarget{ref-Gurung2007}{}
Gurung, R. A., \& Vespia, K. (2007). Looking Good, Teaching Well?
Linking Liking, Looks, and Learning. \emph{Teaching of Psychology},
\emph{34}(1), 5--10.
doi:\href{https://doi.org/10.1080/00986280709336641}{10.1080/00986280709336641}

\hypertarget{ref-Hayes2017}{}
Hayes, A. F. (2017). \emph{Introduction to mediation, moderation, and
conditional process analysis : a regression-based approach}. (T. D.
Little, Ed.) (2nd ed., p. 692). The Guilford Press. Retrieved from
\url{https://www.guilford.com/books/Introduction-to-Mediation-Moderation-and-Conditional-Process-Analysis/Andrew-Hayes/9781462534654}

\hypertarget{ref-HughFeeley2002}{}
Hugh Feeley, T. (2002). Evidence of Halo Effects in Student Evaluations
of Communication Instruction. \emph{Communication Education},
\emph{51}(3), 225--236.
doi:\href{https://doi.org/10.1080/03634520216519}{10.1080/03634520216519}

\hypertarget{ref-Isely2005}{}
Isely, P., \& Singh, H. (2005). Do Higher Grades Lead to Favorable
Student Evaluations? \emph{The Journal of Economic Education},
\emph{36}(1), 29--42.
doi:\href{https://doi.org/10.3200/JECE.36.1.29-42}{10.3200/JECE.36.1.29-42}

\hypertarget{ref-Krautmann1999}{}
Krautmann, A. C., \& Sander, W. (1999). Grades and student evaluations
of teachers. \emph{Economics of Education Review}, \emph{18}(1), 59--63.
doi:\href{https://doi.org/10.1016/S0272-7757(98)00004-1}{10.1016/S0272-7757(98)00004-1}

\hypertarget{ref-Marsh1987}{}
Marsh, H. W. (1987). Students' evaluations of University teaching:
Research findings, methodological issues, and directions for future
research. \emph{International Journal of Educational Research},
\emph{11}(3), 253--388.
doi:\href{https://doi.org/10.1016/0883-0355(87)90001-2}{10.1016/0883-0355(87)90001-2}

\hypertarget{ref-Marsh2000}{}
Marsh, H. W., \& Roche, L. A. (2000). Effects of grading leniency and
low workload on students' evaluations of teaching: Popular myth, bias,
validity, or innocent bystanders? \emph{Journal of Educational
Psychology}, \emph{92}(1), 202--228.
doi:\href{https://doi.org/10.1037/0022-0663.92.1.202}{10.1037/0022-0663.92.1.202}

\hypertarget{ref-Maurer2006}{}
Maurer, T. W. (2006). Cognitive Dissonance or Revenge? Student Grades
and Course Evaluations. \emph{Teaching of Psychology}, \emph{33}(3),
176--179.
doi:\href{https://doi.org/10.1207/s15328023top3303_4}{10.1207/s15328023top3303\_4}

\hypertarget{ref-Miller1985}{}
Miller, A. (1985). A developmental study of the cognitive basis of
performance impairment after failure. \emph{Journal of Personality and
Social Psychology}, \emph{49}(2), 529--538.
doi:\href{https://doi.org/10.1037/0022-3514.49.2.529}{10.1037/0022-3514.49.2.529}

\hypertarget{ref-Miller1989}{}
Miller, A., \& Klein, J. S. (1989). Individual differences in ego value
of academic performance and persistence after failure.
\emph{Contemporary Educational Psychology}, \emph{14}(2), 124--132.
doi:\href{https://doi.org/10.1016/0361-476X(89)90030-1}{10.1016/0361-476X(89)90030-1}

\hypertarget{ref-Palmer2014}{}
Palmer, M. S., Bach, D. J., \& Streifer, A. C. (2014). Measuring the
Promise: A Learning-Focused Syllabus Rubric. \emph{To Improve the
Academy}, \emph{33}(1), 14--36.
doi:\href{https://doi.org/10.1002/tia2.20004}{10.1002/tia2.20004}

\hypertarget{ref-Pinheiro2017}{}
Pinheiro, J., Bates, D., Debroy, S., Sarkar, D., \& Team, R. C. (2017).
nlme: Linear and nonlinear mixed effects models. Retrieved from
\url{https://cran.r-project.org/package=nlme}

\hypertarget{ref-Salmons1993}{}
Salmons, S. D. (1993). The relationship between students' grades and
their evaluation of instructor performance. \emph{Applied H.R.M
Research}, \emph{4}(2), 102--114. Retrieved from
\url{http://psycnet.apa.org/record/2000-14222-002}

\hypertarget{ref-Stanny2017}{}
Stanny, C. J., \& Arruda, J. E. (2017). A comparison of student
evaluations of teaching with online and paper-based administration.
\emph{Scholarship of Teaching and Learning in Psychology}, \emph{3}(3),
198--207.
doi:\href{https://doi.org/10.1037/stl0000087}{10.1037/stl0000087}

\hypertarget{ref-Stanny2015}{}
Stanny, C., Gonzalez, M., \& McGowan, B. (2015). Assessing the culture
of teaching and learning through a syllabus review. \emph{Assessment \&
Evaluation in Higher Education}, \emph{40}(7), 898--913.
doi:\href{https://doi.org/10.1080/02602938.2014.956684}{10.1080/02602938.2014.956684}

\hypertarget{ref-Stark2014}{}
Stark, P., \& Freishtat, R. (2014). An Evaluation of Course Evaluations.
\emph{ScienceOpen Research}.
doi:\href{https://doi.org/10.14293/S2199-1006.1.SOR-EDU.AOFRQA.v1}{10.14293/S2199-1006.1.SOR-EDU.AOFRQA.v1}

\hypertarget{ref-Tabachnick2012}{}
Tabachnick, B. G., \& Fidell, L. S. (2012). \emph{Using multivariate
statistics} (Sixth.). Boston, MA: Pearson.

\hypertarget{ref-Uttl2017}{}
Uttl, B., White, C. A., \& Gonzalez, D. W. (2017). Meta-analysis of
faculty's teaching effectiveness: Student evaluation of teaching ratings
and student learning are not related. \emph{Studies in Educational
Evaluation}, \emph{54}, 22--42.
doi:\href{https://doi.org/10.1016/j.stueduc.2016.08.007}{10.1016/j.stueduc.2016.08.007}

\hypertarget{ref-Weiner1992}{}
Weiner, B. (1992). \emph{Human motivation : metaphors, theories, and
research} (p. 391). Sage.






\end{document}
