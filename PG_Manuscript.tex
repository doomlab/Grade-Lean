\documentclass[man]{apa6}

\usepackage{amssymb,amsmath}
\usepackage{ifxetex,ifluatex}
\usepackage{fixltx2e} % provides \textsubscript
\ifnum 0\ifxetex 1\fi\ifluatex 1\fi=0 % if pdftex
  \usepackage[T1]{fontenc}
  \usepackage[utf8]{inputenc}
\else % if luatex or xelatex
  \ifxetex
    \usepackage{mathspec}
    \usepackage{xltxtra,xunicode}
  \else
    \usepackage{fontspec}
  \fi
  \defaultfontfeatures{Mapping=tex-text,Scale=MatchLowercase}
  \newcommand{\euro}{€}
\fi
% use upquote if available, for straight quotes in verbatim environments
\IfFileExists{upquote.sty}{\usepackage{upquote}}{}
% use microtype if available
\IfFileExists{microtype.sty}{\usepackage{microtype}}{}

% Table formatting
\usepackage{longtable, booktabs}
\usepackage{lscape}
% \usepackage[counterclockwise]{rotating}   % Landscape page setup for large tables
\usepackage{multirow}		% Table styling
\usepackage{tabularx}		% Control Column width
\usepackage[flushleft]{threeparttable}	% Allows for three part tables with a specified notes section
\usepackage{threeparttablex}            % Lets threeparttable work with longtable

% Create new environments so endfloat can handle them
% \newenvironment{ltable}
%   {\begin{landscape}\begin{center}\begin{threeparttable}}
%   {\end{threeparttable}\end{center}\end{landscape}}

\newenvironment{lltable}
  {\begin{landscape}\begin{center}\begin{ThreePartTable}}
  {\end{ThreePartTable}\end{center}\end{landscape}}

  \usepackage{ifthen} % Only add declarations when endfloat package is loaded
  \ifthenelse{\equal{\string man}{\string man}}{%
   \DeclareDelayedFloatFlavor{ThreePartTable}{table} % Make endfloat play with longtable
   % \DeclareDelayedFloatFlavor{ltable}{table} % Make endfloat play with lscape
   \DeclareDelayedFloatFlavor{lltable}{table} % Make endfloat play with lscape & longtable
  }{}%



% The following enables adjusting longtable caption width to table width
% Solution found at http://golatex.de/longtable-mit-caption-so-breit-wie-die-tabelle-t15767.html
\makeatletter
\newcommand\LastLTentrywidth{1em}
\newlength\longtablewidth
\setlength{\longtablewidth}{1in}
\newcommand\getlongtablewidth{%
 \begingroup
  \ifcsname LT@\roman{LT@tables}\endcsname
  \global\longtablewidth=0pt
  \renewcommand\LT@entry[2]{\global\advance\longtablewidth by ##2\relax\gdef\LastLTentrywidth{##2}}%
  \@nameuse{LT@\roman{LT@tables}}%
  \fi
\endgroup}


\ifxetex
  \usepackage[setpagesize=false, % page size defined by xetex
              unicode=false, % unicode breaks when used with xetex
              xetex]{hyperref}
\else
  \usepackage[unicode=true]{hyperref}
\fi
\hypersetup{breaklinks=true,
            pdfauthor={},
            pdftitle={Perceived Grading and Student Evaluation of Instruction},
            colorlinks=true,
            citecolor=blue,
            urlcolor=blue,
            linkcolor=black,
            pdfborder={0 0 0}}
\urlstyle{same}  % don't use monospace font for urls

\setlength{\parindent}{0pt}
%\setlength{\parskip}{0pt plus 0pt minus 0pt}

\setlength{\emergencystretch}{3em}  % prevent overfull lines


% Manuscript styling
\captionsetup{font=singlespacing,justification=justified}
\usepackage{csquotes}
\usepackage{upgreek}

 % Line numbering
  \usepackage{lineno}
  \linenumbers


\usepackage{tikz} % Variable definition to generate author note

% fix for \tightlist problem in pandoc 1.14
\providecommand{\tightlist}{%
  \setlength{\itemsep}{0pt}\setlength{\parskip}{0pt}}

% Essential manuscript parts
  \title{Perceived Grading and Student Evaluation of Instruction}

  \shorttitle{Perceived Grading and Student Evaluation}


  \author{Erin M. Buchanan\textsuperscript{1}, Becca N. Huber\textsuperscript{1}, Arden Miller\textsuperscript{1}, David W. Stockburger\textsuperscript{2}, \& Marshall Beauchamp\textsuperscript{3}}

  % \def\affdep{{"", "", "", "", ""}}%
  % \def\affcity{{"", "", "", "", ""}}%

  \affiliation{
    \vspace{0.5cm}
          \textsuperscript{1} Missouri State University\\
          \textsuperscript{2} US Air Force Academy\\
          \textsuperscript{3} University of Missouri - Kansas City  }

  \authornote{
    A portion of this research was presented at the meeting of the
    Southwestern Psychological Association, April, 2009, San Antonio, TX.
    The authors would like to thank Melissa Fallone for comments on earlier
    drafts and Stephen Martin for his help with restructuring the data.
  }


  \abstract{We analyzed student evaluations for 3,585 classes collected over 20
years to determine stability and evaluate the relationship of perceived
grading to global evaluations, perceived fairness, and appropriateness
of assignments. Using class as the unit of analysis, we found small
evaluation reliability when professors taught the same course in the
same semester, with much weaker correlations for differing courses.
Expected grade and grading related questions correlated with overall
evaluations of courses. Differences in course evaluations on expected
grades, grading questions, and overall grades were found between
full-time faculty and other types of instructors. These findings are
expanded to a model of grading type questions mediating the relationship
between expected grade and overall course evaluations with a moderating
effect of type of instructor.}
  \keywords{Student evaluation, teacher evaluation, perceived grading, reliability \\

    
  }





\usepackage{amsthm}
\newtheorem{theorem}{Theorem}[section]
\newtheorem{lemma}{Lemma}[section]
\theoremstyle{definition}
\newtheorem{definition}{Definition}[section]
\newtheorem{corollary}{Corollary}[section]
\newtheorem{proposition}{Proposition}[section]
\theoremstyle{definition}
\newtheorem{example}{Example}[section]
\theoremstyle{definition}
\newtheorem{exercise}{Exercise}[section]
\theoremstyle{remark}
\newtheorem*{remark}{Remark}
\newtheorem*{solution}{Solution}
\begin{document}

\maketitle

\setcounter{secnumdepth}{0}



Student evaluations of professors have been disputed over time with
regard to validity and reliability. The impact of student evaluations on
professor advancement can be great and often acts as a deciding factor
in professor promotion or demotion, along with the access to certain
funding opportunities, coursework choice, and tenureship. There are
certain variables researched that result in improving evaluations, such
as giving higher grades ({\textbf{???}}; {\textbf{???}};
{\textbf{???}}). Student evaluations have also been found to be
influenced by likability, attractiveness, and dress ({\textbf{???}};
{\textbf{???}}; {\textbf{???}}). Further, 20 years ago, (Neath 1996)
suggests 20 tips in which professors may bolster their evaluations from
students that have no relationship with proven instructional methods or
further learning retention among the student body, such as being a male
professor and only teaching only male students. In more recent research,
({\textbf{???}}) confirms that student evaluations of teaching are
biased against female instructors, and the authors conclude student
evaluations are more representative of the students' grading
expectations and biases rather than an evaluation of objective
instructional methods. All together, these findings elicit the argument
that student evaluations are not necessarily measuring whether the
instructional methods of professors are sound, rather student
evaluations of instruction are measuring whether or not the professor is
likeable among the students with regard to their expectation of their
performance in the classroom, in addition to the instructor meeting
pre-existing biases.

However, some authors ({\textbf{???}}; {\textbf{???}}; {\textbf{???}})
have discovered professors are not able to increase their positive
evaluations by only providing their students with higher grades. We
believe this is due to what we consider the effect of \enquote{perceived
grading}. We operationally define perceived grading as the students'
perceptions of assignment appropriateness, grading fairness, and the
expected course grade at the time the evaluations are being completed.
We believe social psychology theory would support that students with low
perceived grading may reduce cognitive dissonance and engage in ego
defense by giving low evaluations of professors who give them lower
grades ({\textbf{???}}), resulting in decreased validity and reliability
of the proposed construct, professor instruction. We argue that both
social psychology theory and the evidence from student evaluations
supports that higher perceived grading can lead to better student
evaluations of instruction. This theory and evidence from student
evaluation leads us to further posit student evaluations of professors
as biased methods of data collection and irrelevant to the quality of
the instructor and the instructional methods used over the course of a
semester.

Much of the literature on student evaluations involves diverse and
complex analyses (e.g., ({\textbf{???}})) and lacks social-psychological
theoretical guidance on human judgment. To expect that student
evaluations would not be influenced by expected grade would contradict a
long-standing history of social psychology research on cognitive
dissonance, attribution, and ego threat. As we know, failure threatens
the ego {[}({\textbf{???}}); Snyder, Stephan \& Rosenfield, 1978{]} and
motivates us to find rationales to defend the ego. Failing students, or
those performing below personal expectations, would be expected to
defend their ego by attributing low grades to poor teaching or unfair
evaluation practices ({\textbf{???}}). One common strategy involves
diminishing the value of the activity ({\textbf{???}}), which would
result in lowered perceived value of a course.

Similarly, Cognitive Dissonance Theory ({\textbf{???}}) predicts that
people who experience poor performance but perceive themselves as
competent will experience dissonance, of which they can reduce through
negative evaluations of the instruction ({\textbf{???}}). Attribution
research ({\textbf{???}}) also supports the argument that among low
achievement motivation students, failure is associated with external
attributions for cause, and the most plausible external attribution is
the quality of instruction and grading practices. Although arguments
regarding degree of influence are reasonable, the position that they are
not affected is inconsistent with existing and established theory. Thus,
it is not surprising that the majority of faculty perceive student
evaluations to be biased by perceived grading and course choice
({\textbf{???}}).

Considerable research has been conducted in support of widely
distributed evaluation systems. ({\textbf{???}}) reported that in a
study of 9,194 class averages using the Student Instructional Support,
the relationship between expected grades and global ratings was only
.20. He further argued that when variance due to perceived learning
outcomes was regressed from the global evaluation, the effect of
expected grades was eliminated. However, a student's best assessment of
\enquote{perceived learning outcome} is their expected grade, and thus,
these should be highly correlated. When perceived learning is regressed
from the global evaluations, it is not surprising that suppression
effects would eliminate or could even reverse the correlation between
expected grade and global evaluation. In general, there are several
reasons why the relationship of expected grade to global evaluations is
suppressed. For example, faculty ratings are generally very high on
average (i.e.~quality instructors are hired), which restricts variation;
thus, weakening their reliability as a measure of professor attributes.
This restriction in range suppresses correlation.

However, ({\textbf{???}}) provided causal evidence of lowered student
evaluations due to expected grades. In her study of 444 students
completing faculty evaluations at two separate points in a semester,
students who expected to get Fs significantly lowered their evaluations
while students who expected to receive As and Bs significantly raised
their evaluations.

({\textbf{???}}) argued that the individual is also not the proper unit
of analysis because such analyses could suggest false findings related
to individual differences in students. Therefore, he argued the use of
class as the suggested unit of analysis. We agree, both for his
reasoning and because analyses with individual ratings can mask
significant relationships as well (do we have a source for this
claim???). Individual differences in expectancy will attenuate the
correlation less when class average is used as the unit of analysis. To
the extent that the same class average would be expected across all
courses, an assumption we will challenge, the class average for expected
grade is a good measure of perceived grading as an instructor attribute.
Course quality, not individual attributes of students, is what we are
attempting to assess when we are using student evaluations of courses.
Several studies provide support that when class is the unit of analysis,
expected grade is a more significant biasing factor in student
evaluations ({\textbf{???}}; {\textbf{???}}; {\textbf{???}}).

Additionally, ({\textbf{???}}) analyzed 167 psychology classes in a
multiple regression analysis with class as the unit of analysis and
found that the two most significant predictors of instructor ratings
were average grade given by the instructor and instructor status (TA or
rank of faculty). Because of the limited number of classes, the power of
the analysis was limited. However, in addition to the concern regarding
the relationship between grades and global course evaluations, it was
found that TAs were rated more highly than ranked faculty. This finding
raises additional questions on validity student evaluation of
instructional quality. We must either accept that the least trained and
qualified instructors are actually better teachers, or we must believe
this result suggests that student evaluations have given us false
information on the quality of instruction via their perceptions of
grading.

({\textbf{???}}) provided further evidence that using course as a unit
of analysis increased the correlation between expected grade and other
course ratings. Within specific groupings of classes, these correlations
ranged from .23 to .53. Two factors limited the level of their
relationships. First, the classes used were all upper division courses
and graduate courses. Secondly, over 90\% of the students in these
classes expected an A or a B. Consequently, the correlations between
expected grade and global course ratings would be reduced due to the
absence of variation in expected grades. Similarly, ({\textbf{???}})
found a correlation of .35 between average course grade and average
rating of the instructor in 165 classes during a two-year period.
However, these studies did not consider the predictive relationship for
instructors across different courses and semesters, which was one aim of
the current study.

It is pertinent to note that different disciplines and subject areas
have diverse GPA standards, and students have differing grade and
workload expectations in different courses, as well. For example, an
instructor in Anatomy giving a 3.00 GPA might be considered lenient
while an Education instructor giving a 3.25 GPA might be considered hard
(examples for illustration only). To have a valid measure of workload
and leniency factors, correlations should be conducted with varied
teachers of the same course. Further, different populations take courses
in different disciplines, resulting in potential population differences
between anatomy classes and education classes, which could create or
mask findings. Hence, analysis of these correlations within the same
discipline and course would be expected to strengthen the relationship
between expected grades and quality measures, offering more valid
results.

Further, in most studies of student evaluations, reliability is
established through internal consistency reliability. However, this form
of reliability is confounded with halo effects (i.e.~a cognitive bias
that influences ratings based on an overall perception of the person
teaching, rather than the individual components of the course), and
tells only whether the individual responding to the questions is
consistent and reliable(??? do we have a source for halo effects from
internal consistency reliability???). By having many different classes
for the same instructor, we can establish the reliability of ratings
across the same and different courses during the same and different
semesters. As a result, we should be able to deduce if student ratings
can be considered a valid measure of an instructor's teaching skills if
they are or are not able to reliably differentiate instructors within
the same course across different semesters.

If ratings are, in fact, valid measures of instructor attributes, it
should be expected that ratings would have some stability across
semester and specific course taught. If variation were due to instructor
attributes and not the course they are assigned, we would expect ratings
to be most stable across two different courses during the same semester.
We would expect these correlations to decline somewhat for the same
course in a different semester, since faculty members may improve or
decline with experience. But if they are reliable and stable enough to
use in making choices about retention, their stability should be
demonstrated across different semesters, as well. Therefore, in the
current study, we first sought to establish reliability of ratings for
the instructors across courses and semesters.

The current study used data collected over a 20-year period to allow for
more powerful analyses, with such analyses occurring within many
sections of the same course at the same university. After examining
reliability, we sought to show that items on instructor evaluations were
positively correlated for undergraduate and graduate students (???didn't
we want to eliminate graduate students from the analyses because they're
a special population???), demonstrating that overall course evaluations
are related to the perceived grading of the students. We also expected
correlations to be substantially higher than those obtained by previous
researchers who used individual students as their unit of analysis,
since we used the course as the unit of analysis. Next, we examined if
rating differences across these questions were found between types of
instructors compared to full-time faculty, such as teaching-assistants
and per-course faculty. The presumption of university hiring
requirements that include a terminal degree for regular faculty is that
better-trained faculty will be more effective teachers. Therefore, if
student evaluations are a valid measure, better-trained, full-time
faculty should receive higher ratings than per-course instructors and
teaching assistants. However, existing literature appears to contradict
this expectation ({\textbf{???}}). Given these differences, we proposed
and examined a moderated mediation analysis to portray the expected
relationship of the variables across instructor type.

\hypertarget{method}{%
\section{Method}\label{method}}

The archival study was conducted using data from the psychology
department at a large Midwestern public university. We used data from
4313 undergraduate, 397 mixed-level undergraduate, and 687 graduate
psychology classes taught from 1987 to 2016 that were evaluated by
students using the same 15-item instrument. The graduate courses were
excluded from analyses due to the ceiling effects on expected grades.
Faculty followed set procedures in distributing scan forms no more than
two weeks before the conclusion of the semester. A student was assigned
to collect the forms and deliver them to the departmental secretary. The
instructor was required to leave the room while students completed the
forms.

We focused upon the five items, which seemed most pertinent to the
issues of perceived grading and evaluation. We were most interested in
how grades related to global course evaluation and grading/assignment
evaluations. These items were presented with a five-point scale from 1
(\emph{strongly disagree}) to 5 (\emph{strongly agree}):

\begin{verbatim}
1. The overall quality of this course was among the top 20% of those I have taken. 
2. The examinations were representative of the material covered in the assigned readings and class lectures. 
3. The instructor used fair and appropriate methods in the determination of grades. 
4. The assignments and required activities in this class were appropriate. 
5. What grade do you expect to receive in this course? (A = 5, B, C, D, F = 1).
\end{verbatim}

\hypertarget{results}{%
\section{Results}\label{results}}

All data were checked for course coding errors, and type of instructor
was coded as graduate assistant, per-course faculty, instructors, and
tenure-track faculty. This data was considered structured by instructor;
therefore, all analyses below were coded in \emph{R} using the
\emph{nlme} package ({\textbf{???}}) to control for correlated error of
instructor as a random intercept in a multilevel model. The overall
dataset was screened for normality, linearity, homogeneity, and
homoscedasticity using procedures from ({\textbf{???}}). Data generally
met assumptions with a slight skew and some heterogeneity. This data was
not screened for outliers because it was assumed that each score was
entered correctly from student evaluations. The complete set of all
statistics can be found online at \url{http://osf.io/jdpfs}. This page
also includes the manuscript written inline with the statistical
analysis with the \emph{papaja} package ({\textbf{???}}) for interested
researchers/reviewers.

\hypertarget{reliability-of-instructor-scores-done}{%
\subsection{Reliability of Instructor Scores
DONE}\label{reliability-of-instructor-scores-done}}

\begin{table}[tbp]
\begin{center}
\begin{threeparttable}
\caption{\label{tab:rel-table}Correlations for Instructor, Semester, and Course Combinations}
\small{
\begin{tabular}{lllccccc}
\toprule
Instructor & Semester & Course & $b$ & $SE$ & $t$ & $df$ & $p$\\
\midrule
Different Instructor & Different Semester & Different Course & -.001 & .000 & 10144295 & -3.58 & .013\\
Different Instructor & Same Semester & Different Course & .006 & .002 & 152801 & 2.91 & .048\\
Different Instructor & Different Semester & Same Course & .008 & .001 & 517353 & 6.24 & .027\\
Different Instructor & Same Semester & Same Course & .054 & .010 & 6265 & 5.40 & < .001\\
Same Instructor & Different Semester & Different Course & -.038 & .003 & 108849 & -13.13 & < .001\\
Same Instructor & Same Semester & Different Course & .095 & .020 & 1872 & 4.66 & < .001\\
Same Instructor & Different Semester & Same Course & .090 & .004 & 55057 & 21.77 & < .001\\
Same Instructor & Same Semester & Same Course & .446 & .023 & 1401 & 19.63 & < .001\\
\bottomrule
\end{tabular}
}
\end{threeparttable}
\end{center}
\end{table}

Reliability of ratings of instructors can be inferred by the consistency
of ratings across courses and semester, assuming that we infer there is
a stable good/poor instructor attribute and that these multiple
administrations of the same question are multiple assessments of that
attribute. A file was created with all possible course pairings for
every instructor, semester, and course combination. Therefore, this
created eight possible combinations of matching v. no match for
instructor by semester by course. Multilevel models were used to
calculate correlations on each fo the eight combinations controlling for
response size for both courses (i.e., course 1 number of ratings and
course 2 number of ratings) and random intercepts for instructor(s).
Correlations were calculated separately for each question, however, the
overall pattern of the data was the same for each of the eight
combinations, and these were averaged for Table @ref:(tab:rel-table).
The complete set of all correlations can be found online. Because the
large sample size can bias statistical significance, we focused on the
size of the correlations. The correlations were largest for the same
instructor in the same semester and course, followed by the same
instructor in the same semester with a different course and the same
instructor in a different semester with the same course. The first shows
that scores are somewhat reliable (i.e., \emph{r}s \textasciitilde{}
.45) for instructors teaching two or more of the same class at the same
time. The correlations within instructor then drop to \emph{r}s
\textasciitilde{} .09 for the same semester or same course. All other
correlations are nearly zero, with the same semester, same course, and
different instructor as the next largest at \emph{r}s \textasciitilde{}
.05. Given these values are still low for traditional reliability
standards, this results may indicate that student demand characteristics
or course changes impact instructor ratings.

\hypertarget{correlations-of-evaluation-questions-done}{%
\subsection{Correlations of Evaluation Questions
DONE}\label{correlations-of-evaluation-questions-done}}

\begin{table}[tbp]
\begin{center}
\begin{threeparttable}
\caption{\label{tab:correlation-table}t Statistics for Undergraduate Correlations}
\begin{tabular}{lcccccl}
\toprule
Coefficient & \multicolumn{1}{c}{*pr*} & \multicolumn{1}{c}{*b*} & \multicolumn{1}{c}{*SE*} & \multicolumn{1}{c}{*df*} & \multicolumn{1}{c}{*t*} & \multicolumn{1}{c}{*p*}\\
\midrule
Overall to Exams & .637 & .828 & .014 & 4447 & 60.813 & < .001\\
Overall to Fair & .606 & .903 & .016 & 4447 & 57.837 & < .001\\
Overall to Assignments & .675 & .999 & .016 & 4447 & 63.251 & < .001\\
Overall to Grade & .344 & .597 & .022 & 4447 & 27.167 & < .001\\
Exams to Fair & .655 & .751 & .012 & 4447 & 61.387 & < .001\\
Exams to Assignments & .615 & .700 & .014 & 4447 & 50.425 & < .001\\
Exams to Grade & .311 & .416 & .018 & 4447 & 23.066 & < .001\\
Fair to Assignments & .720 & .715 & .011 & 4447 & 63.912 & < .001\\
Fair to Grade & .375 & .438 & .016 & 4447 & 27.865 & < .001\\
Assignments to Grade & .344 & .404 & .015 & 4447 & 26.913 & < .001\\
\bottomrule
\end{tabular}
\end{threeparttable}
\end{center}
\end{table}

Table \ref{tab:correlation-table} presents the inter-correlations for
the five relevant evaluation questions using instructor as a random
intercept in a multilevel model with evaluation sample size as an
adjustor variable. The partial correlation (\emph{pr}) is the
standardized coefficient from the multilevel model analysis between
items while adjusting for sample size and random effects of instructor.
The raw coefficient \emph{b}, standard error, and significance
statistics are also provided. We found class expected grade was related
to class overall rating, exams reflecting the material, grading
fairness, and appropriateness of assignments; however, these partial
correlations were approximately half of all other pairwise correlations.
The correlations between grading related items were high, representing
some consistency in evaluation, as well as the overall course evaluation
to grading questions.

\hypertarget{instructor-status-and-ratings-do-we-want-this}{%
\subsection{Instructor Status and Ratings DO WE WANT
THIS}\label{instructor-status-and-ratings-do-we-want-this}}

We compared teaching assistants, per-course faculty, instructors, and
ranked faculty in undergraduate courses that included evaluations for
all four types of teacher, usually general education classes (i.e.,
Introductory Psychology), required major courses (i.e., Statistics,
Research Methods), and popular electives (i.e., Abnormal Psychology).
This anlaysis included 179 teachers: 49 teaching assistants, 54
per-course instructors, 17 instructors, and 59 full-time faculty who
taught 2744 courses: 266 teaching assistants, 400 per-course
instructors, 354 instructors, and 1724 full-time faculty.

. All comparisons were made against full-time faculty to control for
Type 1 error using a multilevel model with a dummy coded instructor
variable, and dummy coded t values were used to determine which
comparison groups were different from full-time faculty. Overall means
and standard deviations are presented in Table 3, and the complete set
of t value comparisons for these analyses can be found online. As shown
in the Table 3, the ratings of all groups were fairly high, hovering
around 4.00 on a 5.00 point scale, and the expected grade for courses
was approximately a B.

For overall ratings, faculty were found to be rated less highly than
teaching assistants, p = .027, but not significantly different than
per-course faculty (p = .181) or instructors (p = .814). When rating if
exams were representative of course material, full-time faculty were
rated lower than both teaching assistants (p \textless{} .001) and
per-course faculty (p = .047), but were not significantly different than
instructors (p = .740). Full-time faculty were rated as less fair and
appropriate in their grades than teaching assistants (p = .003), while
per-course faculty (p = .128) and instructors (p = .657) had similar
scores to faculty. Teaching assistants were designated to have more
appropriate assignments than faculty (p \textless{} .001), while
per-course (p = .060) and instructors (p = .073) had the same ratings as
faculty on assignments. Finally, faculty showed significantly lower
expected grades than teaching assistants (p \textless{} .001) and
per-course faculty (p = .044), while having similar grades to
instructors (p = .705).

\hypertarget{moderated-mediation}{%
\subsection{Moderated Mediation}\label{moderated-mediation}}

Given the correlations between items and differences between items and
ranked faculty, we proposed a mediation relationship between expected
grade, perceived grading, and overall course grades that varies by
instructor type. Figure 1 demonstrates the predicted relationship
between these variables. We hypothesized that expected course grade
would impact the overall course rating, but this relationship would be
mediated by the perceived grading in the course, which was calculated by
averaging questions about exams, fairness of grading, and assignments.
Therefore, as students expected to earned higher grades (leniency),
their perception and ratings of the grading would increase, thus,
leading to higher overall course scores. This relationship was tested
using traditional and newer approaches to mediation ({\textbf{???}};
Baron \& Kenny, 1986). All categorical interactions were compared to
ranked faculty. Each step of the model is described below. Because
significant interactions were found, we calculated each group separately
(Figure 1) to portray these differences in path coefficients. Tables of
t values for the overall and separated analyses are available at
\url{http://osf.io/jdpfs}.

\hypertarget{c-path}{%
\subsubsection{C Path}\label{c-path}}

First, expected grade was used to predict the overall rating of the
course, along with the interaction of type of instructor and expected
grade. The expected grade positively predicted overall course rating, p
\textless{} .001, wherein higher expected grades was related to higher
overall ratings for the course (b = 0.39). A significant interaction
between type and expected grade rating was found for instructors versus
faculty. In looking at Figure 1, we find that instructors (b = 0.56)
have a stronger relationship between expected grade and overall course
rating than faculty (b = 0.39, interaction p = .020), while per-course
(b = 0.41, interaction p = .621) and teaching assistants (b = 0.71,
interaction p = .068) were not significantly different than faculty on
the c path coefficient.

\hypertarget{a-path}{%
\subsubsection{A Path}\label{a-path}}

Expected grade was then used to predict the average of the grading
related questions, along with the interaction of type of instructor.
Higher expected grades were related to higher ratings of appropriating
grading (b = 0.21, p \textless{} .001), and a significant interaction of
faculty and all three other instructor types emerged: teaching
assistants (p = .001), per-course faculty (p = .001), and instructors (p
\textless{} .001). As seen in Figure 1, faculty (b = 0.21) have a much
weaker relationship between expected grade and average ratings of
grading than teaching assistants (b = 0.55), per-course (b = 0.41), and
instructors (b = 0.45). B and C' Paths. In the final model, expected
grade, average ratings of grading, and the two-way interactions of these
two variables with type were used to predict overall course evaluation.
Average rating of grading was a strong significant predictor of overall
course rating (b = 1.10, p \textless{} .001), indicating that a
perception of fair grading was related positively to overall course
ratings. An interaction between per-course faculty and fair grading
emerged, p \textless{} .001, wherein faculty (b = 1.10) had a less
positive relationship than per-course (b = 1.28), while teaching
assistants (b = 1.37, interaction p = .071) and instructors (b = 1.16,
interaction p = .187) were not significantly different coefficients. The
relationship between expected grade and overall course rating decreased
from the original model (b = 0.16, p \textless{} .001). However, the
interaction between this path and per-course (p \textless{} .001) and
instructors (p = .041) versus faculty was significant, while faculty
versus teaching assistants' paths were not significantly different (p =
.133). Faculty relationship between expected grade and overall course
scoring, while accounting for ratings of grading was stronger (b = 0.16)
than instructors (b = 0.04) and per-course (b = -0.10), but not that of
teaching assistants (b = -0.04).

\#\#\#Mediation Strength

We then analyzed the indirect effects (i.e.~the amount of mediation) for
each type of instructor separately, using both the Aroian version of the
Sobel test (Baron \& Kenny, 1986), as well as bootstrapped samples to
determine the 95\% confidence interval of the mediation (Preacher \&
Hayes, 2008; {\textbf{???}}) because of the criticisms on Sobel. For
confidence interval testing, we ran 5,000 bootstrapped samples examining
the mediation effect and interpreted that the mediation was different
from zero if the confidence interval did not include zero. For teaching
assistants, we found mediation significantly greater than zero, indirect
= 0.74 (SE = 0.14), Z = 5.15, p \textless{} .001, 95\% CI{[}0.48,
1.02{]}. Per-course faculty showed mediation between expected grade and
overall course rating, indirect = 0.52 (SE = 0.09), Z = 6.06, p
\textless{} .001, 95\% CI{[}0.36, 0.73{]}. Instructors showed a similar
indirect mediation effect, indirect = 0.53 (SE = 0.07), Z = 7.31, p
\textless{} .001, 95\% CI{[}0.40, 0.66{]}. Last, faculty showed the
smallest mediation effect, indirect = 0.23 (SE = 0.02), Z = 8.71, p
\textless{} .001, 95\% CI{[}0.19, 0.28{]}, wherein the confidence
interval did not include zero, but also did not overlap with any other
instructor group.

\hypertarget{discussion}{%
\section{Discussion}\label{discussion}}

The findings support the model that student evaluations of Psychology
faculty are related to what one might consider leniency (i.e., overall
average scores of B) in grading through perceptions of assignment
appropriateness, grading fairness, and the expected course grade. This
position is supported both in the strong relationships between expected
grade and global ratings by the evidence that greater training and
experience is related to poorer evaluations, lower expected grades, and
lower relationships between grading and evaluations. Faculty received
lower scores than teaching assistants in every category and often lower
scores than per-course faculty, but not instructors. Mediation analyses
showed that expected grade is positively related to overall course
ratings, although this relationship is mediated by the perceived grading
in the course. Therefore, as students have higher expected grades, the
perceived grading scores increase, and the overall course score also
increases. Moderation of this mediation effect indicated differences in
the strength of the relationships between expected grade, grading
questions, and overall course rating, wherein faculty generally had
weaker relationships between these variables.

Because the study was not experimental, causal conclusions from this
study alone need to be limited. However, ({\textbf{???}}) provides some
evidence of the causal direction of student ratings of instructors and
expected grades. She had 444 students complete faculty evaluations after
3-4 weeks of classes, and again after 13 weeks. Students who expected to
get Fs significantly lowered their evaluations while students who
expected to receive As and Bs significantly raised their evaluations.

It is compelling that the correlations suggest that we can do a better
job of understanding global ratings, perception of exams, fairness, and
appropriateness of assignments based upon what grade students expected
as compared to relating these ratings using ratings for the same course
in a different semester or ratings for a different course in the same
semester for instructor (i.e., correlations between items in the same
semester are higher than reliability estimates across the board). It is
very likely that these correlations with expected grade are suppressed
by the loading of scores at the high end of the scale for course ratings
and expected grade. Generally, evaluation items reflect scores at the
high end of the 1-5 scale (see Table 3) even when items are
intentionally constructed to move evaluators from the ends. The item,
\enquote{The overall quality of this course was among the top 20\% of
those I have taken,} is conspicuously designed to move subjects away
from the top rating. Yet average global ratings remain about a 4.00. The
grade expectation average was approximately 4.00, which relates to a B
average or 3.00 GPA.

One way of establishing convergent validity would be a finding that
better trained and more experienced teachers get higher ratings than
less well trained instructors. If the measure were valid, we would
expect that regular faculty and full time instructors would get higher
ratings than per course faculty and teaching assistants. To argue
otherwise is to challenge the merits of higher education units with a
faculty of professors with doctoral status. If the university were a
researcher powerhouse where faculty research is emphasized over teaching
and graduate assistants are admitted from the highest ranks of
undergraduates, the finding that teaching assistants and per course
faculty get higher ratings might be less of a challenge to the validity
of these ratings. However, the university at which the data were
collected is a non-doctoral program with greater emphasis on teaching
and moderate emphasis on research, and teaching assistants are master's
candidates with less substantial admission expectations than doctoral
programs. Hence, these findings challenge the convergent validity of the
teaching evaluations.

Like most studies in this area, a major limitation is the absence of an
independent measure of learning. Of course, this limitation is based
upon the belief that the goal is to create educated persons, not just
satisfied consumers. Even when common tests are used, these are invalid
if the instructors are aware of the course content. Teachers seeking
high evaluations are able to improve their ratings and scores by
directly addressing the content of the specific test items. ETS now
allows faculty who administer Major Field Tests to access the specific
items which thereby invalidates it as a measure for these purposes.
Ultimately, answering questions about the validity of student
evaluations is a daunting task without such measures.

Evidence suggests that student evaluations are influenced by likability,
attractiveness, and dress ({\textbf{???}}; {\textbf{???}};
{\textbf{???}}) in addition to leniency and low demands
({\textbf{???}}). One must question whether a factor like instructor
warmth, which relates to student evaluation ({\textbf{???}}), is really
fitting to the ultimate purposes of a college education. In a unique
setting where student assignments to courses were random and common
tests were used, ({\textbf{???}}) demonstrated that teaching strategies
that enhanced student evaluations led to poorer performance in
subsequent classes. With the sum of invalid variance from numerous
factors being potentially high, establishment of a high positive
relationship to independent measures of achievement is essential to the
acceptance of student evaluations as a measure of teaching quality.

Perception of the influence of leniency on teacher evaluations is far
more detrimental to the quality of education than the biased evaluations
themselves. It is unlikely that good teachers, even if more challenging,
will get bad evaluations (i.e.~evaluations where the majority of
students rate the course poorly). Good teachers are rarely losing their
positions due to low quality evaluations. But ({\textbf{???}}) found
that faculty perceives evaluations to be biased based upon course
difficulty (72\%), expected grade (68\%), and course workload (60\%). If
one's goal is high merit ratings and teaching awards, and the most
significant factor is student evaluations of teaching, then putting
easier and low-level questions on the test, adding more extra credit,
cutting the project expectations, letting students off the hook for
missing deadlines, and boosting borderline grades would all be likely
strategies for boosting evaluations.

Effective teachers will get positive student ratings even when they have
high expectations and do not inflate grades. But, many excellent
teachers will score below average. It is maladaptive to try to increase
a 3.90 global rating to a 4.10, because it often requires that the
instructor try to emphasize avoidance of the lowest rating (1.00)
because these low ratings in a skewed distribution have in inordinate
influence on the mean. This effort of competing against the norms is
likely to lead to grade inflation and permissiveness for the least
motivated and most negligent students. Some researchers ({\textbf{???}};
{\textbf{???}}) argue that student evaluations of instruction should be
adjusted on the basis of grades assigned. However, there are problems
with such an approach. The regression Betas are likely to differ based
upon course and many other factors. In our research and in research by
DuCettte and Kenney (1982), substantial variation in correlations was
found across different course sets. Establishing valid adjustments would
be problematic at best. Further, such an approach would punish
instructors when they happen to get an unusually intelligent and
motivated class (or teach an honors class) and give students the grades
they deserve. Student evaluations are not a proper motivational factor
for instructors in grade assignment, whether it is to inflate or deflate
grades.

It would seem nearly impossible to eliminate invalid bias in student
ratings of instruction. Yet, they may tell us a teacher is ineffective
when the majority give poor ratings. It is the normative, competitive
use that makes student evaluations of teaching subject to problematic
interpretation. This finding is especially critical in light of recent
research that portrays that student evaluations are largely biased
against female teachers, and that student bias in evaluation is related
to course discipline and student gender ({\textbf{???}}).
({\textbf{???}}) also examine the difficulty in adjusting faculty
evaluation for bias and determined that the complex nature of ratings
makes unbiased evaluation nearly impossible. ({\textbf{???}}) further
explain that evaluations are often negatively related to more objective
measures of teaching effectiveness, and biased additionally by perceived
attractiveness and ethnicity. In line with the current paper, he
suggests dropping overall teaching effectiveness or value of the course
type questions because they are influenced by many variables unrelated
to actual teaching. Last, they suggest the distribution and response
rate of the data are critical information, and this point becomes
particularly important when recent research shows that online
evaluations of teaching experience a large drop in response rates
({\textbf{???}}). Our study contributes to the literature of how student
evaluations are a misleading and unsuccessful measure of teaching
effectiveness, especially focusing on reliability and the impact of
grading on overall questions. We conclude that it may be possible to
manipulate these values by lowering teaching standards, which implies
that high stakes hiring and tenure decisions should probably follow the
advice of ({\textbf{???}}) or ({\textbf{???}}) in implementing teaching
portfolios and syllabus review, particularly because a recent
meta-analysis of student evaluations showed they are unrelated to
student learning ({\textbf{???}}).

\newpage

\hypertarget{references}{%
\section{References}\label{references}}

\setlength{\parindent}{-0.5in}
\setlength{\leftskip}{0.5in}

\hypertarget{refs}{}
\leavevmode\hypertarget{ref-Baron1986}{}%
Baron, R. M., \& Kenny, D. A. (1986). The moderator-mediator variable
distinction in social psychological research: Conceptual, strategic, and
statistical considerations. \emph{Journal of Personality and Social
Psychology}, \emph{51}(6), 1173--1182.
doi:\href{https://doi.org/10.1037//0022-3514.51.6.1173}{10.1037//0022-3514.51.6.1173}






\end{document}
